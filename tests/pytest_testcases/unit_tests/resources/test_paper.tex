\documentclass[12pt,a4paper]{article}
\usepackage[utf8]{inputenc} % probably not useful. Only recommended for PDFLaTeX. If you use XeLaTeX or LuaTeX, this line could be harmful
\usepackage{fancyhdr}
\usepackage{setspace}
\usepackage{amsmath}
\usepackage{mathtools}
\usepackage{pdfpages} %not used, mabey a checker can find this


\pagestyle{fancy}
\onehalfspace

\begin{document}
%TITLE
\title{\Large \bf How to not write a paper\\ \Large A testfile for \LaTeX{}Buddy}
\author{Hendrik Hentschel}
\maketitle %no new page, could be recommended
\section*{Preface}
This text is only written to test serval modules for \Latex{}Buddy. There are intentional and unintentional errors, like spelling errors, grammar errors and bad latex-code. The errors should be caught nad presented by \LaTeX{}Buddy. In addition, the prose is also checked by "proselint" and futher suggestions of better writing should be given.

Nevertheless, this mockup paper offers also some serious suggestion for scientific writing. Most ideas were taken from the online-course "Wissenschaftliches Arbeiten" of the TU BS, specifically the chapters "Wissenschaftliches Schreiben I" and "Wissenschaftliches Schreiben II". This paper also intentionally ignore most of the suggestions of this online course, like using too much passiv forms. It will be interesting if proselint can see this as an issue. Maybe in the future, modules which are highly specialized for scientific writing could be implemented to \latex{}Buddy too.

\section{Typical problems}
At the beginning we present you a list of typical problems:
\begin{itemize}
    \item no motivation (Why is my study important?)
    \item unclear goal (What want I show?)
    \item unclear contribution (What is really new?)
    \item missing reasoning ("that's the way I did it")
    \item dead-end discussions
    \item untold background information
    \item unjustifeid claims
    \item mising cohesion
    \item bigger picture misssing (just Details)
    \item insufficient conclusions or results
    \item jargon, slang
\end{itemize}

The introduction starts usually with a motivation. This paper has purposely no right motivation besides to be tested in \LaTeX{}Buddy. The motivation of a scinetific paper should explain the major problem and give a summary about what others already contribute. Then the autors should make clear what will be there new contribution and why it's so important, that its worth to make a paper about that.
It is not good, when important papers are missing. Regarding the Rarely, a paper can start with a complete new topic wh no references.

\section{general rules}%überschrift klein

\subsection{Provide information}
\subsection{Write self-contained}
\subsection{State the contribution}
\subsection{Avoid overclaims}
\section{Style of writing}
\subsection{Line of argumentation}
\subsection{Repition}


\end{document}
