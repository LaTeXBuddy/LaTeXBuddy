\documentclass[12pt,a4paper]{article}
\usepackage[utf8]{inputenc}
\usepackage{fancyhdr}
\usepackage{setspace}
\usepackage{amsmath}
\usepackage{mathtools}
\usepackage{pdfpages} %nicht benutzt


\pagestyle{fancy}
\onehalfspace

\begin{document}
%TITLE
\title{\Large \bf Wie man kein Paper schreibt\\ \Large Eine Testdatei \LaTeX{}Buddy}
\author{Hendrik Hentschel}
\maketitle %no new page, could be recommended
\section*{Vorwort}
Dieser Text dient nur dazu Module von \Latex{}Buddy zu testen. Er hat absichtiliche und unabsichtliche Fehler, hauptsächlich Rechtschreib- und Gramatikfehler, sowie schlechten LaTeX-Code. Diese Fehler sollen von \LaTeX{}Buddy gefunden werden und angezeigt werden. Außerdem kann für englische Texte die Prosa durch "proselint" gechekct werden. Dann werden weitere Empfehlungen für einen besseren Schreibstil gegeben. Ein derartiges Tool ist bis jetzt nicht für die deutsche Sprache implementiert, aber das kann ja noch kommen.

Nichtsdestotrotz bietet dieser Attrappentext einige ernstgemeinte Vorschläge für wisschenschaftliches Schreiben. Die meissten Ideen stammten aus dem Onlinekurs "Wissenschaftliches Arbeiten" der TU BS, insbesondere aus den Kapiteln "Wissenschaftliches Schreiben I" and "Wissenschaftliches Schreiben II".

\section{Typische Probleme}
Am Anfang zeigen wir Ihnen eine Liste mit tyoischen Problemen:
\begin{itemize}
    \item keine Motivation (Warum ist meine Arbeit wichtig?)
    \item unklare Ziele (Was will ich eigentlich zeigen?)
    \item nicht zu erkennender Beitrag zum bereits bestehenden (Was ist neu?)
    \item fehlender Beitrag ("so habe ichs gemacht")
    \item Sackgassendiskussionen
    \item nicht erwähntes Hintergrundwissen
    \item ungerechtfertigte Behauptungen
    \item fehlender Zusammenhalt
    \item das große Ganze fehlt (nur Details)
    \item ungenügende Rückschlüsse oder Ergebnisse
    \item saloppe Umgangssprache
\end{itemize}

Die Einleitung starten normalerweise mit einer Motivation. Dieses Paper hat absichtlicherweise keinr richtige Motivation außer im \LaTeX{}Buddy getestet zu werden. Die Motivatoin eine wissenschaftlichen Papers sollte das Hauptproblem erläutren und einen Überblick darüber geben, was andere bereits dazu beigetragen haben. Danach sollten die Autoren klar machen, was ihr neuer Beitrag sein wird und warum dies ein Paper wert ist.
It is not good, when important papers are missing. Regarding the Rarely, a paper can start with a complete new topic wh no references.

\section{general rules}%überschrift klein

\subsection{Provide information}
\subsection{Write self-contained}
\subsection{State the contribution}
\subsection{Avoid overclaims}
\section{Style of writing}
\subsection{Line of argumentation}
\subsection{Repition}


\end{document}
