\documentclass[12pt,a4paper]{article}
\usepackage[utf8]{inputenc} % probably not useful. Only recommended for PDFLaTeX. If you use XeLaTeX or LuaTeX, this line could be harmful
\usepackage{fancyhdr}
\usepackage{setspace}
\usepackage{amsmath}
\usepackage{mathtools}
\usepackage{pdfpages} %not used, mabey a checker can find this


\pagestyle{fancy}
\onehalfspace

\begin{document}
%TITLE
\title{\Large \bf How to not write a paper\\ \Large A testfile for \LaTeX{}Buddy}
\author{Hendrik Hentschel}
\maketitle %no new page, could be recommended
\section*{Preface}
This is a corrected version of a mock up paper, I wrote to test serval modules for \Latex{}Buddy. The original text has intentional and unintentional errors, like spelling errors, grammar errors and bad latex-code. The text of this file should show much less errors, but won't be completly errorfree.

Nevertheless, this mockup paper offers also some serious suggestion for scientific writing. Most ideas were taken from the online-course "Wissenschaftliches Arbeiten" of the TU BS, specifically the chapters "Wissenschaftliches Schreiben I" and "Wissenschaftliches Schreiben II".

\section{Typical problems}
At the beginning we present you a list of typical problems:
\begin{itemize}
    \item missing motivation (Why is it important?)
    \item unclear goal, unclear contribution
    \item missing reasoning ("that's the way I did it")
    \item dead-end discussions, unused background
    \item unjustified claims
    \item missing cohesion
    \item bigger picture missing (just details)
    \item missing conclusions or results
    \item jargon, background missing
    \item related work missing
\end{itemize}

The introduction starts usually with a motivation. This paper has purposely no right motivation besides to be tested in \LaTeX{}Buddy. The motivation of a scinetific paper should explain the major problem and give a summary about what others already contribute. Then the autors should make clear what will be there new contribution and why it's so important, that its worth to make a paper about that.
It is not good, when important papers are missing. Regarding the Rarely, a paper can start with a complete new topic wh no references.

\section{general rules}%überschrift klein

\subsection{Provide information}
\subsection{Write self-contained}
\subsection{State the contribution}
\subsection{Avoid overclaims}
\section{Style of writing}
\subsection{Line of argumentation}
\subsection{Repition}


\end{document}
